
%**************************************************************
% Acronimi
%**************************************************************
\renewcommand{\acronymname}{Acronimi e abbreviazioni}

\newacronym[description={\glslink{apig}{Application Program Interface}}]
    {api}{API}{Application Program Interface}
		
\newacronym[description={\glslink{restg}{Representational State Transfer}}]
		{rest}{REST}{Representational State Transfer}

\newacronym[description={\glslink{umlg}{Unified Modeling Language}}]
    {uml}{UML}{Unified Modeling Language}

%**************************************************************
% Glossario
%**************************************************************
%\renewcommand{\glossaryname}{Glossario}


\newglossaryentry{apig}
{
    name=\glslink{api}{API},
    text=Application Program Interface,
    sort=api,
    description={in informatica con il termine \emph{Application Programming Interface API} (ing. interfaccia di programmazione di un'applicazione) si indica ogni insieme di procedure disponibili al programmatore, di solito raggruppate a formare un set di strumenti specifici per l'espletamento di un determinato compito all'interno di un certo programma. La finalità è ottenere un'astrazione, di solito tra l'hardware e il programmatore o tra software a basso e quello ad alto livello semplificando così il lavoro di programmazione}
}

\newglossaryentry{backend}
{
		name=\glslink{backend}{Backend},
		sort=backend,
		description={un'applicazione o un programma backend serve indirettamente come supporto ai servizi del frontend, solitamente essendo vicino alla risorsa richiesta o avendo la capacità di comunicare con data risorsa. L'applicazione backend può interagire direttamente con il frontend oppure, forse più tipicamente, è un programma chiamato da un programma intermediario che si inserisce tra le attività del frontend e del backend.
		}
}

\newglossaryentry{frontend}
{
		name=\glslink{frontend}{Frontend},
		sort=frontend,
		description={nel campo della progettazione software il front end è la parte di un sistema software che gestisce l'interazione con l'utente o con sistemi esterni che producono dati di ingresso (es. interfaccia utente con un form).}
}

\newglossaryentry{restg}
{
		name=\glslink{rest}{REST},
		text=REST,
		sort=rest,
		description={REST si riferisce ad un insieme di principi di architetture di rete, i quali delineano come le risorse sono definite e indirizzate. Il termine è spesso usato nel senso di descrivere ogni semplice interfaccia che trasmette dati su HTTP senza un livello opzionale.}
}
		
\newglossaryentry{umlg}
{
    name=\glslink{uml}{UML},
    text=UML,
    sort=uml,
    description={in ingegneria del software \emph{UML, Unified Modeling Language} (ing. linguaggio di modellazione unificato) è un linguaggio di modellazione e specifica basato sul paradigma object-oriented. L'\emph{UML} svolge un'importantissima funzione di ``lingua franca'' nella comunità della progettazione e programmazione a oggetti. Gran parte della letteratura di settore usa tale linguaggio per descrivere soluzioni analitiche e progettuali in modo sintetico e comprensibile a un vasto pubblico}
}
