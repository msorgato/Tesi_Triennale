
%**************************************************************
% Acronimi
%**************************************************************
\renewcommand{\acronymname}{Acronimi e abbreviazioni}

\newacronym[description={\glslink{apig}{Application Program Interface}}]
    {api}{API}{Application Program Interface}

\newacronym[description={\glslink{crudg}{Create Read Update Delete}}]
	{crud}{CRUD}{Create Read Update Delete}

\newacronym[description={\glslink{hateoasg}{Hypermedia as the Engine of Application State}}]
	{hateoas}{HATEOAS}{Hypermedia as the Engine of Application State}

\newacronym[description={\glslink{ictg}{Information and Communication Technology}}]
	{ict}{ICT}{Information and Communication Technology}

\newacronym[description={\glslink{ideg}{Integrated Development Environment}}]
	{ide}{IDE}{Integrated Development Environment}

\newacronym[description={\glslink{mvcg}{Model View Controller}}]
	{mvc}{MVC}{Model View Controller}
		
\newacronym[description={\glslink{restg}{Representational State Transfer}}]
	{rest}{REST}{Representational State Transfer}

\newacronym[description={\glslink{umlg}{Unified Modeling Language}}]
    {uml}{UML}{Unified Modeling Language}

%**************************************************************
% Glossario
%**************************************************************
%\renewcommand{\glossaryname}{Glossario}

\newglossaryentry{angularjs}
{
	name=\glslink{angularjs}{AngularJS}
	sort=angularjs,
	description={comunemente noto con il nome "Angular", è un framework open source mantenuto da \emph{Google} e da una comunità di sviluppatori individuali e corporazioni. Il suo fine principale è la creazione di Single Page Application, semplificando la codifica e il testing. L'architettura di base di Angular è il pattern \emph{MVC}.}
}

\newglossaryentry{apig}
{
    name=\glslink{api}{API},
    text=Application Program Interface,
    sort=api,
    description={in informatica con il termine \emph{Application Programming Interface API} (ing. interfaccia di programmazione di un'applicazione) si indica ogni insieme di procedure disponibili al programmatore, di solito raggruppate a formare un set di strumenti specifici per l'espletamento di un determinato compito all'interno di un certo programma. La finalità è ottenere un'astrazione, di solito tra l'hardware e il programmatore o tra software a basso e quello ad alto livello semplificando così il lavoro di programmazione.}
}

\newglossaryentry{backend}
{
	name=\glslink{backend}{Backend},
	sort=backend,
	description={un'applicazione o un programma backend serve indirettamente come supporto ai servizi del frontend, solitamente essendo vicino alla risorsa richiesta o avendo la capacità di comunicare con data risorsa. L'applicazione backend può interagire direttamente con il frontend oppure, forse più tipicamente, è un programma chiamato da un programma intermediario che si inserisce tra le attività del frontend e del backend.}
}

\newglossaryentry{compliance}
{
	name=\glslink{compliance}{Compliance},
	sort=compliance,
	description={atto di essere in allineamento con linee guida, regolamentazioni e/o legislazioni.}
}

\newglossaryentry{crudg}
{
	name=\glslink{crud}{CRUD},
	text=Create Read Update Delete,
	sort=crud,
	description={con la sigla \emph{CRUD} si rappresenta l'insieme delle quattro operazioni basilari su dei dati persistenti, ovvero creazione, lettura, modifica ed eliminazione.}
}

\newglossaryentry{frontend}
{
	name=\glslink{frontend}{Frontend},
	sort=frontend,
	description={nel campo della progettazione software il front end è la parte di un sistema software che gestisce l'interazione con l'utente o con sistemi esterni che producono dati di ingresso (es. interfaccia utente con un form).}
}

\newglossaryentry{governance}
{
	name=\glslink{governance}{Governance},
	sort=governance,
	description={L'IT Governance è responsabilità diretta del consiglio di amministrazione e del management esecutivo. \`{E} parte integrante della governance aziendale ed è costituita dalla direzione, dalla struttura organizzativa e dai processi in grado di assicurare che l'IT sostenga ed estenda gli obiettivi e le strategie dell'organizzazione.}
}

\newglossaryentry{hateoasg}
{
	name=\glslink{hateoas}{HATEOAS},
	text=HATEOAS,
	sort=hateoas,
	description={vincolo delle applicazioni basate su architettura REST che le distingue dalla maggior parte delle altre applicazioni web. Il principio consiste in un client che interagisce con un'applicazione web esclusivamente attraverso gli ipermedia forniti dinamicamente dai server dell'applicazione stessa. Un client REST non necessita così di nessuna conoscenza a priori per interagire con una particolare applicazione o server che vada oltre la normale comprensione degli ipermedia.}
}

\newglossaryentry{ictg}
{
	name=\glslink{ict}{ICT},
	text=ICT,
	sort=ict,
	description={acronimo di \emph{Information and Communication Technology}, tradotto in Tecnologie dell’informazione e della comunicazione. Sono l'insieme dei metodi e delle tecnologie che realizzano i sistemi di trasmissione, ricezione ed elaborazione di informazioni.}
}

\newglossaryentry{ideg}
{
	name=\glslink{ide}{IDE}
	text=IDE,
	sort=ide,
	description={si traduce in italiano come Ambiente di Sviluppo Integrato. Un IDE è un software che, in fase di programmazione, aiuta i programmatori nello sviluppo del codice sorgente di un programma. Spesso l'IDE aiuta lo sviluppatore segnalando errori di sintassi del codice direttamente in fase di scrittura, oltre a tutta una serie di strumenti e funzionalità di supporto alla fase di sviluppo e debugging.}
}

\newglossaryentry{mvcg}
{
	name=\glslink{mvc}{MVC},
	text=MVC,
	sort=mvc,
	description={pattern architetturale software utilizzato nell'implementazione di interfacce utente. Divide l'applicazione software in tre parti interconnesse, così da separare la rappresentazione interna delle informazioni dal modo in cui tali informazioni sono presentate all'utente.}
}


%LE NOZIONI TEORICHE VANNO MESSE SEMPRE QUI NEL GLOSSARIO O VA CREATA UNA SEZIONE APPOSITA NELL'APPENDICE??

\newglossaryentry{repository}
{
	name=\glslink{repository}{Repository},
	sort=repository,
	description={struttura di supporto su cui un software può essere progettato e realizzato. Alla base di un framework sono sempre presenti delle librerie di codice utilizzabili con uno o più linguaggi di programmazione; esse sono spesso corredate da una serie di strumenti di supporto allo sviluppo software, o altri strumenti ideati per aumentare la velocità di sviluppo del prodotto finito. Lo scopo di un framework è quello di far risparmiare allo sviluppatore la riscrittura di codice già scritto precedentemente per fini simili.}
}

\newglossaryentry{restg}
{
	name=\glslink{rest}{REST},
	text=REST,
	sort=rest,
	description={REST si riferisce ad un insieme di principi di architetture di rete, i quali delineano come le risorse sono definite e indirizzate. Il termine è spesso usato nel senso di descrivere ogni semplice interfaccia che trasmette dati su HTTP senza un livello opzionale.}
}

\newglossaryentry{spring}
{
	name=\glslink{spring}{Spring},
	sort=spring,
	description={framework open source per lo sviluppo di applicazioni su piattaforma Java. Spring fornisce un modello comprensivo di programmazione e configurazione per applicazioni \emph{enterprise} basate sulla piattaforma Java.}
}
		
\newglossaryentry{umlg}
{
    name=\glslink{uml}{UML},
    text=UML,
    sort=uml,
    description={in ingegneria del software \emph{UML, Unified Modeling Language} (ing. linguaggio di modellazione unificato) è un linguaggio di modellazione e specifica basato sul paradigma object-oriented. L'\emph{UML} svolge un'importantissima funzione di ``lingua franca'' nella comunità della progettazione e programmazione a oggetti. Gran parte della letteratura di settore usa tale linguaggio per descrivere soluzioni analitiche e progettuali in modo sintetico e comprensibile a un vasto pubblico}
}
