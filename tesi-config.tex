%**************************************************************
% file contenente le impostazioni della tesi
%**************************************************************

%**************************************************************
% Frontespizio
%**************************************************************
\newcommand{\myName}{Mattia Sorgato}                            % autore
\newcommand{\myTitle}{Sviluppo test driven di un front-end con AngularJS}                              % titolo dello stage
\newcommand{\myDegree}{Tesi di laurea triennale}                % tipo di tesi
\newcommand{\myUni}{Università degli Studi di Padova}           % università
\newcommand{\myFaculty}{Corso di Laurea in Informatica}         % facoltà
\newcommand{\myDepartment}{Dipartimento di Matematica}          % dipartimento
\newcommand{\myProf}{Mauro Conti}                               % relatore
\newcommand{\myLocation}{Padova}                                % dove
\newcommand{\myAA}{2014-2015}                                   % anno accademico
\newcommand{\myTime}{Jun 2015}                                  % quando


%**************************************************************
% Impostazioni di impaginazione
% see: http://wwwcdf.pd.infn.it/AppuntiLinux/a2547.htm
%**************************************************************

\setlength{\parindent}{14pt}   % larghezza rientro della prima riga
\setlength{\parskip}{0pt}   % distanza tra i paragrafi


%**************************************************************
% Impostazioni di biblatex
%**************************************************************
\bibliography{bibliografia} % database di biblatex 

\defbibheading{bibliography}
{
    \cleardoublepage
    \phantomsection 
    \addcontentsline{toc}{chapter}{\bibname}
    \chapter*{\bibname\markboth{\bibname}{\bibname}}
}

\setlength\bibitemsep{1.5\itemsep} % spazio tra entry

\DeclareBibliographyCategory{opere}
\DeclareBibliographyCategory{web}

\addtocategory{opere}{womak:lean-thinking}
\addtocategory{opere}{governance-institute:it-governance}
\addtocategory{opere}{atlassian:jira}
\addtocategory{web}{site:agile-manifesto}
\addtocategory{web}{site:kanban}
\addtocategory{web}{site:q}

\defbibheading{opere}{\section*{Riferimenti bibliografici}}
\defbibheading{web}{\section*{Siti Web consultati}}


%**************************************************************
% Impostazioni di caption
%**************************************************************
\captionsetup{
    tableposition=top,
    figureposition=bottom,
    font=small,
    format=hang,
    labelfont=bf
}

%**************************************************************
% Impostazioni di glossaries
%**************************************************************

%**************************************************************
% Acronimi
%**************************************************************
\renewcommand{\acronymname}{Acronimi e abbreviazioni}

\newacronym[description={\glslink{apig}{Application Program Interface}}]
    {api}{API}{Application Program Interface}
		
\newacronym[description={\glslink{cdng}{Content Delivery Network}}]
		{cdn}{CDN}{Content Delivery Network}

\newacronym[description={\glslink{crudg}{Create Read Update Delete}}]
	{crud}{CRUD}{Create Read Update Delete}
	
\newacronym[description={\glslink{cssg}{Cascading Style Sheet}}]
	{css}{CSS}{Cascading Style Sheet}
	
\newacronym[description={\glslink{e2eg}{End-To-End}}]
	{e2e}{E2E}{End-To-End}

\newacronym[description={\glslink{hateoasg}{Hypermedia as the Engine of Application State}}]
	{hateoas}{HATEOAS}{Hypermedia as the Engine of Application State}

\newacronym[description={\glslink{htmlg}{HyperText Markup Language}}]
	{html}{HTML}{HyperText Markup Language}
	
\newacronym[description={\glslink{httpg}{HyperText Transfer Protocol}}]
	{http}{HTTP}{HyperText Transfer Protocol}

\newacronym[description={\glslink{ictg}{Information and Communication Technology}}]
	{ict}{ICT}{Information and Communication Technology}

\newacronym[description={\glslink{ideg}{Integrated Development Environment}}]
	{ide}{IDE}{Integrated Development Environment}

\newacronym[description={\glslink{mvcg}{Model View Controller}}]
	{mvc}{MVC}{Model View Controller}

\newacronym[description={\glslink{npmg}{Node Package Manager}}]
	{npm}{NPM}{Node Package Manager}
		
\newacronym[description={\glslink{restg}{Representational State Transfer}}]
	{rest}{REST}{Representational State Transfer}
	
\newacronym[description={\glslink{soapg}{Simple Object Access Protocol}}]
  {soap}{SOAP}{Simple Object Access Protocol}
	
\newacronym[description={\glslink{spag}{Single Page Application}}]
	{spa}{SPA}{Single Page Application}

\newacronym[description={\glslink{umlg}{Unified Modeling Language}}]
    {uml}{UML}{Unified Modeling Language}

%**************************************************************
% Glossario
%**************************************************************
%\renewcommand{\glossaryname}{Glossario}

%\newglossaryentry{angularjs}
%{
%	name=\glslink{angularjs}{AngularJS}
%	sort=angularjs,
%	description={comunemente noto con il nome "Angular", è un framework open source mantenuto da %
%\emph{Google} e da una comunità di sviluppatori individuali e corporazioni. Il suo fine principale è la creazione di Single Page Application, semplificando la codifica e il testing. L'architettura di base di Angular è il pattern \emph{MVC}.}
%}

\newglossaryentry{apig}
{
    name=\glslink{api}{API},
    text=Application Program Interface,
    sort=api,
    description={in informatica con il termine \emph{Application Programming Interface API} (ing. interfaccia di programmazione di un'applicazione) si indica ogni insieme di procedure disponibili al programmatore, di solito raggruppate a formare un set di strumenti specifici per l'espletamento di un determinato compito all'interno di un certo programma. La finalità è ottenere un'astrazione, di solito tra l'hardware e il programmatore o tra software a basso e quello ad alto livello semplificando così il lavoro di programmazione.}
}

\newglossaryentry{backend}
{
	name=\glslink{backend}{Backend},
	sort=backend,
	description={un'applicazione o un programma backend serve indirettamente come supporto ai servizi del frontend, solitamente essendo vicino alla risorsa richiesta o avendo la capacità di comunicare con data risorsa. L'applicazione backend può interagire direttamente con il frontend oppure, forse più tipicamente, è un programma chiamato da un programma intermediario che si inserisce tra le attività del frontend e del backend.}
}

\newglossaryentry{cdng}
{
	name=\glslink{cdn}{CDN},
	text=Content Delivery Network,
	sort=cdn,
	description={sistema di computer collegati in rete attraverso Internet, che collaborano in maniera trasparente, sotto forma di sistema distribuito, per distribuire contenuti (specialmente contenuti multimediali di grandi dimensioni in termini di banda) agli utenti finali ed erogare servizi e file di generi diversi.}
}

\newglossaryentry{compliance}
{
	name=\glslink{compliance}{Compliance},
	sort=compliance,
	description={atto di essere in allineamento con linee guida, regolamentazioni e/o legislazioni.}
}

\newglossaryentry{crudg}
{
	name=\glslink{crud}{CRUD},
	text=Create Read Update Delete,
	sort=crud,
	description={con la sigla \emph{CRUD} si rappresenta l'insieme delle quattro operazioni basilari su dei dati persistenti, ovvero creazione, lettura, modifica ed eliminazione.}
}

\newglossaryentry{cssg}
{
	name=\glslink{css}{CSS},
	text=Cascading Style Sheet,
	sort=css,
	description={linguaggio usato per definire la formattazione di documenti HTML, XHTML e XML e relative pagine web. Le regole per comporre il CSS sono contenute in un insieme di direttive (Recommendations) emanate a partire dal 1996 dal W3C.\\
L'introduzione del CSS si è resa necessaria per separare i contenuti dalla formattazione e permettere una programmazione più chiara e facile da utilizzare, sia per gli autori delle pagine HTML che per gli utenti, garantendo contemporaneamente anche il riuso di codice ed una sua più facile manutenibilità.}
}

\newglossaryentry{e2eg}
{
	name=\glslink{e2e}{E2E},
	text=End-To-End,
	sort=ete,
	description={acronimo di \emph{End-To-End}, rappresenta quella categoria di test che si prefigge l'obiettivo di testare il comportamento di un utente su un sistema, usualmente automatizzando gli input utente previsti ed esaminando che i risultati attesi siano rispettati.}
}

\newglossaryentry{frontend}
{
	name=\glslink{frontend}{Frontend},
	sort=frontend,
	description={nel campo della progettazione software il front end è la parte di un sistema software che gestisce l'interazione con l'utente o con sistemi esterni che producono dati di ingresso (es. interfaccia utente con un form).}
}

\newglossaryentry{governance}
{
	name=\glslink{governance}{Governance},
	sort=governance,
	description={L'IT Governance è responsabilità diretta del consiglio di amministrazione e del management esecutivo. \`{E} parte integrante della governance aziendale ed è costituita dalla direzione, dalla struttura organizzativa e dai processi in grado di assicurare che l'IT sostenga ed estenda gli obiettivi e le strategie dell'organizzazione.}
}

\newglossaryentry{hateoasg}
{
	name=\glslink{hateoas}{HATEOAS},
	text=HATEOAS,
	sort=hateoas,
	description={vincolo delle applicazioni basate su architettura REST che le distingue dalla maggior parte delle altre applicazioni web. Il principio consiste in un client che interagisce con un'applicazione web esclusivamente attraverso gli ipermedia forniti dinamicamente dai server dell'applicazione stessa. Un client REST non necessita così di nessuna conoscenza a priori per interagire con una particolare applicazione o server che vada oltre la normale comprensione degli ipermedia.}
}

\newglossaryentry{htmlg}
{
	name=\glslink{html}{HTML},
	text=HTML,
	sort=html,
	description={linguaggio di markup solitamente usato per la formattazione e impaginazione di documenti ipertestuali disponibili nel World Wide Web sotto forma di pagine web.}
}

\newglossaryentry{httpg}
{
	name=\glslink{http}{HTTP},
	text=HTTP,
	sort=http,
	description={tradotto in \emph{protocollo di trasferimento di un ipertesto}, è usato come principale protocollo per la trasmissione d'informazioni sul web.}
}

\newglossaryentry{ictg}
{
	name=\glslink{ict}{ICT},
	text=ICT,
	sort=ict,
	description={acronimo di \emph{Information and Communication Technology}, tradotto in Tecnologie dell’informazione e della comunicazione. Sono l'insieme dei metodi e delle tecnologie che realizzano i sistemi di trasmissione, ricezione ed elaborazione di informazioni.}
}

\newglossaryentry{ideg}
{
	name=\glslink{ide}{IDE}
	text=IDE,
	sort=ide,
	description={si traduce in italiano come Ambiente di Sviluppo Integrato. Un IDE è un software che, in fase di programmazione, aiuta i programmatori nello sviluppo del codice sorgente di un programma. Spesso l'IDE aiuta lo sviluppatore segnalando errori di sintassi del codice direttamente in fase di scrittura, oltre a tutta una serie di strumenti e funzionalità di supporto alla fase di sviluppo e debugging.}
}

\newglossaryentry{mvcg}
{
	name=\glslink{mvc}{MVC},
	text=MVC,
	sort=mvc,
	description={pattern architetturale software utilizzato nell'implementazione di interfacce utente. Divide l'applicazione software in tre parti interconnesse, così da separare la rappresentazione interna delle informazioni dal modo in cui tali informazioni sono presentate all'utente.}
}

\newglossaryentry{npmg}
{
	name=\glslink{npm}{NPM},
	text=NPM,
	sort=npm,
	description={package manager di JavaScript, usato di default da Node.js. Una volta installato Node.js, è possibile utilizzare anche NPM, dato che viene distribuito assieme a suddetto framework. Al suo interno si trovano tutti i maggiori framework ed utilities che si basano sul linguaggio JavaScript.}
}

\newglossaryentry{repository}
{
	name=\glslink{repository}{Repository},
	sort=repository,
	description={struttura di supporto su cui un software può essere progettato e realizzato. Alla base di un framework sono sempre presenti delle librerie di codice utilizzabili con uno o più linguaggi di programmazione; esse sono spesso corredate da una serie di strumenti di supporto allo sviluppo software, o altri strumenti ideati per aumentare la velocità di sviluppo del prodotto finito. Lo scopo di un framework è quello di far risparmiare allo sviluppatore la riscrittura di codice già scritto precedentemente per fini simili.}
}

\newglossaryentry{restg}
{
	name=\glslink{rest}{REST},
	text=REST,
	sort=rest,
	description={REST si riferisce ad un insieme di principi di architetture di rete, i quali delineano come le risorse sono definite e indirizzate. Il termine è spesso usato nel senso di descrivere ogni semplice interfaccia che trasmette dati su HTTP senza un livello opzionale.}
}

\newglossaryentry{soapg}
{
  name=\glslink{soap}{SOAP},
	text=SOAP,
	sort=soap,
	description={acronimo di \emph{Simple Object Access Protocol}, è un protocollo leggero per lo scambio di messaggi tra componenti software, tipicamente nella forma di componentistica software. La parola object manifesta che l'uso del protocollo dovrebbe effettuarsi secondo il paradigma della programmazione orientata agli oggetti.}
}

\newglossaryentry{spag}
{
	name=\glslink{spa}{SPA},
	text=SPA,
	sort=spa,
	description={una Single Page (Web) Application è un'applicazione web o semplicemente un sito che si propone di realizzare un comportamento più espressivo rispetto ad un'applicazione desktop. In una \emph{SPA}, il codice necessario viene caricato al caricamento della pagina e tutti i contenuti vengono visualizzati dinamicamente, mentre tutte le comunicazioni con il server vengono nascoste all'utente.}
}

\newglossaryentry{spring}
{
	name=\glslink{spring}{Spring},
	sort=spring,
	description={framework open source per lo sviluppo di applicazioni su piattaforma Java. Spring fornisce un modello comprensivo di programmazione e configurazione per applicazioni \emph{enterprise} basate sulla piattaforma Java.}
}

\newglossaryentry{stub}
{
	name=\glslink{stub}{Stub},
	sort=stub,
	description={porzione di codice utilizzata in sostituzione di altre funzionalità software. Uno stub può simulare il comportamento di codice esistente e temporaneo sostituto di codice ancora da sviluppare. Gli stub sono perciò molto utili durante il porting di software, l'elaborazione distribuita e in generale durante lo sviluppo di software e il software testing.}
}
		
\newglossaryentry{umlg}
{
    name=\glslink{uml}{UML},
    text=UML,
    sort=uml,
    description={in ingegneria del software \emph{UML, Unified Modeling Language} (ing. linguaggio di modellazione unificato) è un linguaggio di modellazione e specifica basato sul paradigma object-oriented. L'\emph{UML} svolge un'importantissima funzione di ``lingua franca'' nella comunità della progettazione e programmazione a oggetti. Gran parte della letteratura di settore usa tale linguaggio per descrivere soluzioni analitiche e progettuali in modo sintetico e comprensibile a un vasto pubblico}
}
 % database di termini
\makeglossaries


%**************************************************************
% Impostazioni di graphicx
%**************************************************************
\graphicspath{{immagini/}} % cartella dove sono riposte le immagini


%**************************************************************
% Impostazioni di hyperref
%**************************************************************
\hypersetup{
    %hyperfootnotes=false,
    %pdfpagelabels,
    %draft,	% = elimina tutti i link (utile per stampe in bianco e nero)
    colorlinks=true,
    linktocpage=true,
    pdfstartpage=1,
    pdfstartview=FitV,
    % decommenta la riga seguente per avere link in nero (per esempio per la stampa in bianco e nero)
    %colorlinks=false, linktocpage=false, pdfborder={0 0 0}, pdfstartpage=1, pdfstartview=FitV,
    breaklinks=true,
    pdfpagemode=UseNone,
    pageanchor=true,
    pdfpagemode=UseOutlines,
    plainpages=false,
    bookmarksnumbered,
    bookmarksopen=true,
    bookmarksopenlevel=1,
    hypertexnames=true,
    pdfhighlight=/O,
    %nesting=true,
    %frenchlinks,
    urlcolor=webbrown,
    linkcolor=RoyalBlue,
    citecolor=webgreen,
    %pagecolor=RoyalBlue,
    %urlcolor=Black, linkcolor=Black, citecolor=Black, %pagecolor=Black,
    pdftitle={\myTitle},
    pdfauthor={\textcopyright\ \myName, \myUni, \myFaculty},
    pdfsubject={},
    pdfkeywords={},
    pdfcreator={pdfLaTeX},
    pdfproducer={LaTeX}
}

%**************************************************************
% Impostazioni di itemize
%**************************************************************
\renewcommand{\labelitemi}{$\ast$}

%\renewcommand{\labelitemi}{$\bullet$}
%\renewcommand{\labelitemii}{$\cdot$}
%\renewcommand{\labelitemiii}{$\diamond$}
%\renewcommand{\labelitemiv}{$\ast$}


%**************************************************************
% Impostazioni di listings
%**************************************************************
\lstset{
    language=[LaTeX]Tex,%C++,
    keywordstyle=\color{RoyalBlue}, %\bfseries,
    basicstyle=\small\ttfamily,
    %identifierstyle=\color{NavyBlue},
    commentstyle=\color{Green}\ttfamily,
    stringstyle=\rmfamily,
    numbers=none, %left,%
    numberstyle=\scriptsize, %\tiny
    stepnumber=5,
    numbersep=8pt,
    showstringspaces=false,
    breaklines=true,
    frameround=ftff,
    frame=single
} 


%**************************************************************
% Impostazioni di xcolor
%**************************************************************
\definecolor{webgreen}{rgb}{0,.5,0}
\definecolor{webbrown}{rgb}{.6,0,0}


%**************************************************************
% Altro
%**************************************************************

\newcommand{\omissis}{[\dots\negthinspace]} % produce [...]

% eccezioni all'algoritmo di sillabazione
\hyphenation
{
    ma-cro-istru-zio-ne
    gi-ral-din
}

\newcommand{\sectionname}{sezione}
\addto\captionsitalian{\renewcommand{\figurename}{figura}
                       \renewcommand{\tablename}{tabella}}

\newcommand{\glsfirstoccur}{\ap{{[g]}}}

\newcommand{\intro}[1]{\emph{\textsf{#1}}}

%**************************************************************
% Environment per ``rischi''
%**************************************************************
\newcounter{riskcounter}                % define a counter
\setcounter{riskcounter}{0}             % set the counter to some initial value

%%%% Parameters
% #1: Title
\newenvironment{risk}[1]{
    \refstepcounter{riskcounter}        % increment counter
    \par \noindent                      % start new paragraph
    \textbf{\arabic{riskcounter}. #1}   % display the title before the 
                                        % content of the environment is displayed 
}{
    \par\medskip
}

\newcommand{\riskname}{Rischio}

\newcommand{\riskdescription}[1]{\textbf{\\Descrizione:} #1.}

\newcommand{\risksolution}[1]{\textbf{\\Soluzione:} #1.}

%**************************************************************
% Environment per ``API''
%**************************************************************

%%%% Parameters
% #1: Title
\newenvironment{api}[1]{
    \textbf{#1}   % display the title before the 
                                        % content of the environment is displayed 
}{
    \par\medskip
}

\newcommand{\apiurl}[1]{\textbf{\\URL:} #1.}

\newcommand{\apimethod}[1]{\textbf{\\Metodo:} #1.}

\newcommand{\apireq}[1]{\textbf{\\Contenuto della richiesta:} #1.}

\newcommand{\apisolution}[1]{\textbf{\\Soluzione:} #1.}

%**************************************************************
% Environment per ``use case''
%**************************************************************
\newcounter{usecasecounter}             % define a counter
\setcounter{usecasecounter}{0}          % set the counter to some initial value

%%%% Parameters
% #1: ID
% #2: Nome
\newenvironment{usecase}[2]{
    \renewcommand{\theusecasecounter}{\usecasename #1}  % this is where the display of 
                                                        % the counter is overwritten/modified
    \refstepcounter{usecasecounter}             % increment counter
    \vspace{10pt}
    \par \noindent                              % start new paragraph
    {\large \textbf{\usecasename #1: #2}}       % display the title before the 
                                                % content of the environment is displayed 
    \medskip
}{
    \medskip
}

\newcommand{\usecasename}{UC}

\newcommand{\usecaseactors}[1]{\textbf{\\Attori Principali:} #1. \vspace{4pt}}
\newcommand{\usecasepre}[1]{\textbf{\\Precondizioni:} #1. \vspace{4pt}}
\newcommand{\usecasedesc}[1]{\textbf{\\Descrizione:} #1. \vspace{4pt}}
\newcommand{\usecasepost}[1]{\textbf{\\Postcondizioni:} #1. \vspace{4pt}}
\newcommand{\usecasealt}[1]{\textbf{\\Scenario Alternativo:} #1. \vspace{4pt}}

%**************************************************************
% Environment per ``namespace description''
%**************************************************************

\newenvironment{namespacedesc}{
    \vspace{10pt}
    \par \noindent                              % start new paragraph
    \begin{description} 
}{
    \end{description}
    \medskip
}

\newcommand{\classdesc}[2]{\item[\textbf{#1:}] #2}
