% !TEX encoding = UTF-8
% !TEX TS-program = pdflatex
% !TEX root = ../tesi.tex
% !TEX spellcheck = it-IT

%**************************************************************
\chapter{Progettazione e codifica}
\label{cap:progettazione-codifica}
%**************************************************************

\section{Progettazione Architetturale}

\subsection{Front-end REST}
La struttura base dell'applicativo SkillMatrix è un \gls{front-end} \gls{rest}. Un \gls{front-end} di quest tipo ha diversi vantaggi, come la modularità e portabilità che si può ottenere da client a server.\\
Con un tipo di architettura come questo, infatti, il client ed il server rimangono fortemente separati, limitando le chiamate a dei singoli punti di accesso del server. Questa modularità consente una maggiore manutenibilità di entrambe le parti, garantendo allo stesso modo delle interfacce comuni per i client che vogliono comunicare con il server \gls{rest}.\\
Per creare un client \gls{rest}, si ha bisogno di:
\begin{itemize}
	\item un client \gls{http};
	\item le \gls{uri} delle risorse a cui vogliamo accedere;
	\item la capacità di interpretare il formato delle rappresentazione delle risorse.
\end{itemize}
In definitiva, il server mette a disposizione delle risorse e le operazioni con cui è possibile manipolare tali risorse. Solitamente, e per quanto riguarda SkillMatrix, le operazioni che vengono "pubblicizzate" da un server con questa architettura sono le classiche \gls{crud}.\\
Trattandosi di un client che dialoga con il server attraverso protocollo \gls{http}, le richieste da utilizzare per gestire le risorse tramite operazioni \gls{crud} saranno i verbi \gls{http} standard, ovvero:
\begin{itemize}
	\item GET;
	\item POST;
	\item PUT;
	\item DELETE.
\end{itemize}
Rispettivamente, questi verbi consentono di visualizzare i dati della risorsa, di modificarli, di inserire una nuova risorsa o di cancellarla. Ovviamente, a seconda della risorsa e dei permessi del client, certe operazioni possono essere possibili o no (ad esempio, un utente senza privilegi di amministrazione può non avere il permesso di eliminare una risorsa).\\
In questo tipo di architettura, il client deve utilizzare questi verbi \gls{http} per comunicare con il server. Le risposte del server sono riassunte in uno specifico codice di stato, il quale esprime se l'operazione è andata a buon fine, se ci sono stati degli errori di trasmissione o semplicemente trasmette ulteriori informazioni al client. Tali codici di errore sono fondamentali per il client, il quale deve modificare il suo stato interno in base al codice ritornato in risposta dal server. Per fare un esempio, il tipico comportamento di un client che ha richiesto la visualizzazione di una risorsa è, in caso di successo, l'effettivo caricamento delle informazioni della risorsa in una pagina; in caso di errore, ad esempio una risorsa non presente, il client deve sapersi adattare a tale errore e mostrare visivamente quale errore si è presentato, per informare l'utente.

%**************************************************************

\section{Architettura di AngularJS}
AngularJS è stato il framework maggiormente utilizzato in questo stage e mi ha consentito di implementare l'intero progetto agilmente.\\
Alla base del framework, è collocato il design pattern \gls{mvc}, leggermente modificato per adattarsi alle funzionalità di AngularJS. Il design pattern che ne risulta è qualcosa di più flessibile del classico \gls{mvc}, consentendo agli sviluppatori una maggior libertà di utilizzo.\\
Ovviamente ci sono delle direttive e delle \emph{best practice} consigliate, soprattutto se si intende creare un \gls{front-end} davvero \gls{rest}-ful.

\subsection{Model}
In AngularJS, il "`\emph{model}"' è realizzato con precise strutture dati. Innanzitutto, la logica di business dell'applicazione si colloca nei cosiddetti \emph{servizi}. I servizi sono dei \emph{singleton} istanziati con tecnica \emph{lazy}, ovvero alla prima invocazione. In queste strutture è collocata tutta la logica che è indipendente dalla \emph{view}, puntando al suo massimo riutilizzo.\\
Differentemente dai controller, le \emph{Factory} dei servizi vengono richiamate dal sistema di \emph{Dependendcy Injection}. Questo avviene passando come argomento al construttore di una struttura il nome del servizio da richiamare; sarà poi compito dell'\emph{injector} di AngularJS richiamare la corretta \emph{factory} del servizio richiesto.\\
I servizi vegono utilizzati inoltre (e soprattutto) per gestire i dati sensibili dell'utenza e dei vari oggetti presenti nel \gls{back-end}. Sono loro, infatti, gli esecutori materiali delle richieste \gls{rest} al \gls{back-end}, grazie alle direttive apposite, come \textbf{\$http} e \textbf{\$resource}; i dati risultanti verranno poi forniti alle viste, se necessario, passando per gli adeguati controller.

\begin{figure}[!h] 
    \centering 
    \includegraphics[width=0.7\columnwidth]{concepts-module-service} 
    \caption{Esempio di legame tra vista-controller-servizio}
\end{figure}

\subsection{View}
Le viste realizzate con AngularJS sono essenzialmente pagine web con delle funzionalità di \emph{markup} aggiuntive. All'interno di una pagina scritta in codice \gls{html}, si possono utilizzare delle direttive di AngularJS che verranno interpretate prima di renderizzare la pagina. Tramite queste direttive è possibile visualizzare dati del modello, modificarli in \emph{real time} grazie al \emph{Two-way data binding} richiamare funzioni di uno specifico controller e molto altro.\\
La classica dicitura di AngularJS per identificare un'espressione da interpretare si ottiene racchiudendo l'espressione tra due parentesi graffe: 
\begin{verbatim}
{{ espressione }}
\end{verbatim}
All'interno di questa dicitura, ogni espressione viene valutata e il suo risultato viene renderizzato nella pagina web. Se, ad esempio, nel \emph{template} \gls{html} si inserisce un'operazione matematica all'interno delle doppie graffe, una volta renderizzata la pagina il risultato dell'operazione (se possibile) verrà mostrato al posto dell'espressione.\\
Oltre a questa funzionalità, AngularJS mette a disposizione una grande varietà di direttive per manipolare il \gls{dom}, come iteratori per scorrere insiemi di dati o filtri da applicare ad una collezione per limitarne la visualizzazione.

\begin{figure}[!h] 
    \centering 
    \includegraphics[width=0.9\columnwidth]{concepts-databinding1} 
    \caption{Esempio di Template di una vista e Data Binding}
\end{figure}

\subsection{Controller}
I controller in AngularJS sono ciò che implementa la logica applicativa. Vengono legati al \emph{template} di una pagina tramite la direttiva \textbf{ng-controller} posta all'interno di un elemento \gls{html}. Così facendo, si invoca il costruttore del controller designato; lo \emph{scope} del controller appena creato è l'elemento in cui è stato dichiarato e gli eventuali nodi figli dell'elemento stesso.\\
Il principale scopo dei controller è di esporre funzionalità alle espressioni ed alle direttive utilizzate all'interno delle pagine web. Questo si ottiene aggiungendo metodi e proprietà all'oggetto \textbf{\$scope} che verrà condiviso con la vista. Questo oggetto in particolare permette di realizzare il cosiddetto \emph{view-model}, ovvero quella parte del \emph{model} che verrà presentata alla \emph{view}. Inoltre, tutte le proprietà associate al controller saranno rese disponibili al \emph{template}, nello \emph{scope} che compete al controller.

\begin{figure}[!h] 
    \centering 
    \includegraphics[width=0.75\columnwidth]{concepts-databinding2} 
    \caption{Esempio di legame tra template e Controller}
\end{figure}

\subsection{Two-way Data Binding}
Una funzionalità importante che AngularJS espone è il cosiddetto \emph{Two Way Data Binding}. Si parla di \emph{legame doppio tra dati} quando una variabile del modello è legata ad un elemento che può cambiare ed al contempo mostrare il contenuto della variabile stessa. In una vista di AngularJS, ogniqualvolta un elemento che applica il \emph{Two Way Data Binding} viene modificato, il corrispondente campo nel modello viene notificato e aggiornato correttamente.\\
In AngularJS, si usa la direttiva \textbf{ng-model} per legare una variabile del modello ad un elemento \gls{html} che può sia mostrare il suo valore, che modificarlo.

\begin{figure}[H] 
    \centering 
    \includegraphics[width=0.8\columnwidth]{two_way_data_binding} 
    \caption{Doppio legame tra vista e modello di AngularJS}
\end{figure}

\subsection{Dependency Injection}
\emph{Dependency Injection} è un design pattern creato per gestire e risolvere le varie dipendenze tra più moduli software, e viene utilizzato all'interno di questo particale \emph{framework} JavaScript.\\
La responsabilità di fornire le varie componenti richieste dalle dipendenze è demandata al sottosistema di gestione delle iniziezioni di AngularJS. Ogni volta che si dichiara una componente, viene registrata al modulo corrispondente; in questo modo, il sistema di gestione delle dipendenze può riconoscere la componente ed usarla dove è richiesto.\\
Quando è richiesta una dipendenza, l'iniettore controlla il nome con cui è stata richiesta e la cerca nelle dipendenze registrate nel modulo, per iniettarla dove richiesto, se presente.

\begin{figure}[!h] 
    \centering 
    \includegraphics[width=0.7\columnwidth]{concepts-module-injector} 
    \caption{Esempio di configurazione di una factory per l'iniettore}
\end{figure}

%**************************************************************

\section{Definizione delle API REST}
Questa sezione presenta la lista delle \gls{api} concordate per le richieste \gls{rest} al \gls{back-end}. Ogni definizione comprende l'\gls{uri} di riferimento, il tipo di richiesta e i valori di ritorno possibili.\\
Nelle \gls{uri} delle richieste, la dicitura \textbf{vx} sta ad indicare il numero di versione della richiesta, mentre l'espressione \textbf{:variabile} sta ad indicare una porzione variabile dell'\gls{uri}, utilizzata da AngularJS.

\subsection{Informazioni utente}

\subsubsection{Lettura}

\paragraph{URL}
/api/vx/users/:username
\paragraph{Metodo}
GET
\paragraph{Contenuto della richiesta}
nessuno
\paragraph{Risposte}
\begin{itemize}
	\item[200] la risposta contiene il \gls{json} compatto con le proprietà utente;
	\item[404] se l'utente non è presente;
	\item[403] se l'utente non è autorizzato ad accedere all'informazione.
\end{itemize}


\subsubsection{Invio di dati per l'autenticazione}
\paragraph{URL}
/api/vx/login
\paragraph{Metodo}
POST
\paragraph{Contenuto della richiesta}
file \gls{json} con la seguente formattazione:
\begin{verbatim}
{
  'username': 'some_username',
  'password': 'some_password'
}
\end{verbatim}
\paragraph{Risposte}
\begin{itemize}
	\item[200] la risposta contiene il \gls{json} compatto con le proprietà utente, più un \emph{token} utilizzato per comporre l'\emph{header} di autenticazione;
	\item[400] se gli \emph{header} sono invalidi oppure se il contenuto della richiesta non è corretto;
	\item[403] se l'autenticazione non è andata a buon fine.
\end{itemize}

\subsubsection{Creazione di un nuovo utente}
\paragraph{URL}
/api/vx/users
\paragraph{Metodo}
PUT
\paragraph{Contenuto della richiesta}
file \gls{json} compatto contenente username ed email
\paragraph{Risposte}
\begin{itemize}
	\item[201] la risposta contiene l'\emph{id} della proprietà creata;
	\item[400] se gli \emph{header} sono invalidi oppure se il contenuto della richiesta non è corretto;
	\item[403] se l'utente non è autorizzato alla creazione di un nuovo utente.
\end{itemize}

\subsubsection{Modifica dei dati utente}
\paragraph{URL}
/api/vx/users/:username
\paragraph{Metodo}
PATCH
\paragraph{Contenuto della richiesta}
file \gls{json} contenente le modifiche da applicare all'utente designato
\paragraph{Risposte}
\begin{itemize}
	\item[204] corpo vuoto;
	\item[400] se gli \emph{header} sono invalidi oppure se il contenuto della richiesta non è corretto;
	\item[403] se l'utente non è autorizzato a modificare l'entità.
\end{itemize}

\subsection{Esperienze professionali}

\subsubsection{Lettura}
\paragraph{URL}
/api/vx/users/:username/experiences
\paragraph{Metodo}
GET
\paragraph{Contenuto della richiesta}
nessuno
\paragraph{Risposte}
\begin{itemize}
	\item[200] la risposta contiene il \gls{json} con l'\emph{array} di esperienze;
	\item[404] se la risorsa non è presente;
	\item[403] se l'utente non è autorizzato ad accedere all'informazione.
\end{itemize}


\subsubsection{Inserimento di una nuova esperienza}
\paragraph{URL}
/api/vx/users/:username/experiences
\paragraph{Metodo}
PUT
\paragraph{Contenuto della richiesta}
file \gls{json} con la seguente formattazione:
\begin{verbatim}
{
  'role': 'ruolo_aziendale',
  'company': 'nome_azienda',
  'managerName': 'nome_responsabile',
  'managerMail': 'mail_responsabile',
  'beginDate': 'data_ISO',
  'endDate': 'data_ISO',
  'description': "breve descrizione dell'esperienza lavorativa",
  'area': 'area_competenza',
  'technologies': 'stack_utilizzati'
}
\end{verbatim}
\paragraph{Risposte}
\begin{itemize}
	\item[201] la risposta contiene il \gls{json} dell'esperienza appena creata;
	\item[400] se gli \emph{header} sono invalidi oppure se il contenuto della richiesta non è corretto;
	\item[403] se l'utente non è autorizzato a creare una nuova esperienza.
\end{itemize}

\subsubsection{Modifica dei dati di un'esperienza}
\paragraph{URL}
/api/vx/users/:username/esperiences/:experience
\paragraph{Metodo}
PATCH
\paragraph{Contenuto della richiesta}
file \gls{json} contenente le modifiche da applicare all'esperienza dell'utente
\paragraph{Risposte}
\begin{itemize}
	\item[204] corpo vuoto;
	\item[400] se gli \emph{header} sono invalidi oppure se il contenuto della richiesta non è corretto;
	\item[403] se l'utente non è autorizzato a modificare l'entità.
\end{itemize}


\subsection{Titoli di studio ed abilitazioni professionali}

\subsubsection{Lettura}
\paragraph{URL}
/api/vx/users/:username/qualifications
\paragraph{Metodo}
GET
\paragraph{Contenuto della richiesta}
nessuno
\paragraph{Risposte}
\begin{itemize}
	\item[200] la risposta contiene il \gls{json} con l'\emph{array} di qualificazioni;
	\item[404] se la risorsa non è presente;
	\item[403] se l'utente non è autorizzato ad accedere all'informazione.
\end{itemize}


\subsubsection{Inserimento di una nuova qualificazione}
\paragraph{URL}
/api/vx/users/:username/qualifications
\paragraph{Metodo}
PUT
\paragraph{Contenuto della richiesta}
file \gls{json} con una tra le seguenti formattazioni:
\begin{itemize}
\item se la qualificazione in questione non è un \emph{workshop}, ovvero non è un prodotto:
\begin{verbatim}
{
  'type': 'tipo_certificazione',
  'authority': 'ente_di_rilascio',
  'address': 'recapito_ente',
  'grade': 'grado_certificazione',
  'earnDate': 'data_ISO'
}
\end{verbatim}
\item se la qualificazione in questione è un \emph{workshop}:
\begin{verbatim}
{
  'type': 'workshop',
  'authority': 'ente_di_rilascio',
  'address': 'recapito_ente',
  'grade': 'grado_certificazione',
  'earnDate': 'data_ISO',
  'vendor': 'committente',
  'product': 'nome_workshop',
  'expireDate': 'data_ISO'
}
\end{verbatim}
\end{itemize}
\paragraph{Risposte}
\begin{itemize}
	\item[201] la risposta contiene il \gls{json} della qualificazione appena creata;
	\item[400] se gli \emph{header} sono invalidi oppure se il contenuto della richiesta non è corretto;
	\item[403] se l'utente non è autorizzato a creare una nuova qualificazione.
\end{itemize}

\subsubsection{Modifica dei dati di una qualificazione}
\paragraph{URL}
/api/vx/users/:username/qualifications/:qualification
\paragraph{Metodo}
PATCH
\paragraph{Contenuto della richiesta}
file \gls{json} contenente le modifiche da applicare alla qualificazione dell'utente
\paragraph{Risposte}
\begin{itemize}
	\item[204] corpo vuoto;
	\item[400] se gli \emph{header} sono invalidi oppure se il contenuto della richiesta non è corretto;
	\item[403] se l'utente non è autorizzato a modificare l'entità.
\end{itemize}

\subsection{Skill}

\subsubsection{Lettura}
\paragraph{URL}
/api/vx/users/:username/skills
\paragraph{Metodo}
GET
\paragraph{Contenuto della richiesta}
nessuno
\paragraph{Risposte}
\begin{itemize}
	\item[200] la risposta contiene il \gls{json} con l'\emph{array} di \emph{skill};
	\item[404] se la risorsa non è presente;
	\item[403] se l'utente non è autorizzato ad accedere all'informazione.
\end{itemize}


\subsubsection{Inserimento di una nuova skill}
\paragraph{URL}
/api/vx/users/:username/skills
\paragraph{Metodo}
PUT
\paragraph{Contenuto della richiesta}
file \gls{json} con la seguente formattazione:
\begin{verbatim}
{
  'area': 'area_skill',
  'scope': 'ambito_skill',
  'domain': 'dominio_skill',
  'competence': 'array_versioni_dominio',
  'level': Junior/Intermediate/Senior
}
\end{verbatim}
\paragraph{Risposte}
\begin{itemize}
	\item[201] la risposta contiene il \gls{json} della \emph{skill} appena creata;
	\item[400] se gli \emph{header} sono invalidi oppure se il contenuto della richiesta non è corretto;
	\item[403] se l'utente non è autorizzato a creare una nuova \emph{skill}.
\end{itemize}

\subsubsection{Modifica dei dati di una skill}
\paragraph{URL}
/api/vx/users/:username/skills/:skill
\paragraph{Metodo}
PATCH
\paragraph{Contenuto della richiesta}
file \gls{json} contenente le modifiche da applicare alla \emph{skill} dell'utente
\paragraph{Risposte}
\begin{itemize}
	\item[204] corpo vuoto;
	\item[400] se gli \emph{header} sono invalidi oppure se il contenuto della richiesta non è corretto;
	\item[403] se l'utente non è autorizzato a modificare l'entità.
\end{itemize}

\subsection{Progetti associati}

\subsubsection{Lettura lista dei progetti associati all'utente}
\paragraph{URL}
/api/vx/users/:username/projects
\paragraph{Metodo}
GET
\paragraph{Contenuto della richiesta}
nessuno
\paragraph{Risposte}
\begin{itemize}
	\item[200] la risposta contiene il \gls{json} con l'\emph{array} di nomi dei progetti associati;
	\item[404] se la risorsa non è presente;
	\item[403] se l'utente non è autorizzato ad accedere all'informazione.
\end{itemize}

\subsubsection{Lettura delle informazioni di un progetto}
\paragraph{URL}
/api/vx/users/:username/projects/:projectname
\paragraph{Metodo}
GET
\paragraph{Contenuto della richiesta}
nessuno
\paragraph{Risposte}
\begin{itemize}
	\item[200] la risposta contiene il \gls{json} con le proprietà del progetto richiesto;
	\item[404] se la risorsa non è presente;
	\item[403] se l'utente non è autorizzato ad accedere all'informazione.
\end{itemize}

\subsubsection{Inserimento di un nuovo progetto}
\paragraph{URL}
/api/vx/users/:username/projects/:projectname
\paragraph{Metodo}
PUT
\paragraph{Contenuto della richiesta}
file \gls{json} con la seguente formattazione:
\begin{verbatim}
{
  'projectTitle': 'titolo_progetto',
  'description': 'breve descrizione del progetto',
  'client': 'committente',
  'beginDate': 'data_ISO',
  'endDate': 'data_ISO',
  'area': 'area_progetto',
  'technologies': 'array di tecnologie',
  'team': 'array di sviluppatori'
}
\end{verbatim}
\paragraph{Risposte}
\begin{itemize}
	\item[201] la risposta contiene il \gls{json} del progetto appena creato;
	\item[400] se gli \emph{header} sono invalidi oppure se il contenuto della richiesta non è corretto;
	\item[403] se l'utente non è autorizzato a creare un nuovo progetto.
\end{itemize}

\subsubsection{Modifica dei dati di un progetto}
\paragraph{URL}
/api/vx/users/:username/projects/:projectname
\paragraph{Metodo}
PATCH
\paragraph{Contenuto della richiesta}
file \gls{json} contenente le modifiche da applicare al progetto
\paragraph{Risposte}
\begin{itemize}
	\item[204] corpo vuoto;
	\item[400] se gli \emph{header} sono invalidi oppure se il contenuto della richiesta non è corretto;
	\item[403] se l'utente non è autorizzato a modificare l'entità.
\end{itemize}

%**************************************************************


\section{Stub Back-end}
Il progetto di stage ha avuto come oggetto la realizzazione di un \gls{front-end} gestionale, senza però avere al contempo un \gls{back-end} funzionante. Per le verifiche funzionali, quindi, ho avuto bisogno di realizzare un \gls{back-end} fittizio, il quale mi consentisse di effettuare chiamate alle \gls{api} concordate con il Responsabile di Progetto per il \gls{back-end} vero e proprio.\\
Per realizzare questo \gls{back-end} sono ricorso a delle feature fornite dalle librerie di AngularJS adibite al testing di unità. Durante il testing delle funzionalità di una componente, le richieste ad un server funzionante vengono sostituite con una chiamata ad hoc che deve rispettare certe precondizioni e che restituisce dei dati prestabiliti. Il concetto del \gls{back-end} di \gls{stub} che ho utilizzato è lo stesso. In particolare, grazie all'utilizzo di espressioni regolari per filtrare le chiamate \gls{rest}, sono stato in grado di intercettare tali chiamate e di farle gestire al mio \gls{back-end} ad hoc, con risposte prestabilite.\\
Tutto ciò è stato possibile gran parte grazie al servizio \textbf{\$httpBackend} di AngularJS, il quale contiene metodi di intercettazione di chiamate \gls{rest} e di gestione di richieste e risposte (e relativi \emph{header}) \gls{http}. Il frammento sottostante è un esempio di una chiamata \gls{http} gestita dal \gls{back-end} di \gls{stub}:

\begin{verbatim}
$httpBackend.whenGET(/\/api\/\d.\d\/users\/\w*\/qualification/)
.respond(function(method, url, data, headers) {
  console.log('Received these data:', method, url, data, headers);
  var getStatus,
   getData = {},
   getHeaders = headers;
  if(headers.Authorization == "Bearer XXX") {
    getStatus = 200;
    getData = {
      qualifications: [
        {
          type: 'Workshop',
          authority: 'W3C',
          address: 'Silicon Valley, 42',
          grade: 'Senior',
          earnDate: new Date(2015, 4, 20).toISOString(),
          vendor: 'RFC',
          product: 'HTML6',
          expireDate: new Date(2050, 4, 20).toISOString()
        },
        {
          type: 'Certificazione',
          authority: 'Mongo',
          address: 'MongoDB avenue, 8080',
          grade: 'Senior',
          earnDate: new Date(2014, 7, 8).toISOString()
        },
        {
          type: 'Corso',
          authority: 'AngularJS',
          address: '1600 Amphitheatre Parkway Mountain View, CA 94043',
          grade: 'Junior',
          earnDate: new Date(2015, 5, 20).toISOString()
        }
      ]
    };
  }
  else
    getStatus = 401;
  return [getStatus, getData, getHeaders];
});
\end{verbatim} 
%Oltre a permettermi di controllare l'effettivo funzionamento dell'applicazione, un \gls{backend} di %questo tipo mi consente inoltre di verificare 

%**************************************************************

\section{Tecnologie e strumenti}
\label{sec:tecnologie-strumenti}

Di seguito viene data una panoramica delle tecnologie e strumenti utilizzati.

\subsection*{Tecnologia 1}
Descrizione Tecnologia 1.

\subsection*{Tecnologia 2}
Descrizione Tecnologia 2

%**************************************************************
\section{Ciclo di vita del software}
\label{sec:ciclo-vita-software}

%**************************************************************
\section{Progettazione}
\label{sec:progettazione}

\subsubsection{Namespace 1} %**************************
Descrizione namespace 1.

\begin{namespacedesc}
    \classdesc{Classe 1}{Descrizione classe 1}
    \classdesc{Classe 2}{Descrizione classe 2}
\end{namespacedesc}


%**************************************************************
\section{Design Pattern utilizzati}

%**************************************************************
\section{Codifica}
A causa del ciclo di vita \emph{Agile} adottato nello sviluppo di questo progetto, la progettazione e la codifica sono andate di pari passo. La suddivisione del lavoro in \emph{user stories} e l'utilizzo della \emph{kanban} hanno reso molto più disciplinata l'applicazione della metodologia \emph{Agile}, mantenendo comunque la flessibilità e gli aspetti positivi di questo ciclo di vita software. Infatti, \emph{Agile} aumenta molto l'interazione tra proponente e sviluppatore, stimolando l'acquisizione di \emph{feedback} rapidi, cosa molto importante soprattutto in un contesto di stage, a mio avviso.

\subsection{Callback e Promesse JavaScript}
Una delle peculiarità di JavaScript è la gestione delle funzioni. Infatti, in questo linguaggio e nei suoi framework, le funzioni sono considerate degli \emph{oggetti} veri e propri. Ciò consente ad un programmatore alcune operazioni particolari con le funzioni, come ad esempio il passaggio di una funzione come argomento di un'altra funzione.\\
Da questo principio nascono le \emph{callback}, ovvero l'atto di richiamare all'interno di una funzione, un'altra funzione passata come argomento. Ad esempio:
\begin{verbatim}
function calcola(func, arg1, arg2) {
  return func(arg1, arg2);
}
\end{verbatim}
Nell'esempio soprastante, la funzione \emph{func} richiamata all'interno di \emph{calcola}, è stata passata come argomento ed utilizza come propri argomenti \emph{arg1} ed \emph{arg2}. Questa proprietà permette anche, ad esempio, di richiamare selettivamente una \emph{callback} piuttosto che un'altra, o cambiandone gli argomenti a seconda degli stati dell'esecuzione.\\
Le \emph{callback} sono state utilizzate estensivamente anche nel campo della programmazione asincrona e della gestione di eventi. Infatti, una \emph{callback} può essere invocata alla fine di un'acquisizione dati, per gestire i dati ritornati, a prescindere dal momento in cui questi saranno disponibili. Il caso tipico di esempio è una chiamata \gls{ajax}, dove la \emph{callback} viene passata come parametro alla chiamata e viene invocata al termine dell'acquisizione dei dati dalla rete.\\
Purtroppo, però, le \emph{callback} presentano alcuni aspetti negativi:
\begin{itemize}
	\item scarsa leggibilità del codice e conseguente difficoltà di manutenzione;
	\item difficoltà di composizione delle \emph{callback} e conseguente sincronizzazione delle chiamate;
	\item difficoltà di gestione di errori all'interno delle \emph{callback}, soprattutto se queste sono delle \emph{funzioni anonime}.
\end{itemize}
Da questa serie di inconvenienti propri dell'utilizzo delle \emph{callback}, sono state ideate le \emph{promesse}. Questi oggetti rappresentano il risultato di una chiamata di una funzione asincrona, talvolta chiamate con il nome di \emph{future}, \emph{delay} o \emph{deferred}; tutte queste diciture garantiscono che la \emph{promessa} restituisca il dato non appena questo è disponibile.\\
Una \emph{promessa} può trovarsi in uno dei seguenti stati:
\begin{itemize}
	\item \emph{pending}, quando il risultato della chiamata asincrona non è ancora disponibile;
	\item \emph{resolved}, quando la chiamata asincrona ha prodotto un risultato valido;
	\item \emph{rejected}, se la chiamata asincrona non ha prodotto un risultato, anche a causa di un errore da gestire.
\end{itemize}
Oltre a rendere il codice molto più leggibile, le \emph{promesse} consentono di ottenere una maggiore chiarezza nella programmazione e una migliore gestione degli errori che si possono verificare. Ecco un esempio di utilizzo di una \emph{promessa} in una chiamata \gls{ajax}:
\begin{verbatim}
var getUser = function() {
  var deferred = Q.defer();
  $.get("/users", [id: "12345"], deferred.resolve);
  return deferred.promise;
}
 
getUser().then(function(user) {
  console.log("Nome utente: " + user.Name);
});
\end{verbatim}
Nell'esempio soprastante, viene prima creato l'oggetto che rappresenta la \emph{promessa} legata alla chiamata \gls{ajax} dell'utente con \emph{id} "12345", la quale \emph{promessa} viene ritornata dalla funzione. L'invocazione successiva di \emph{getUser()} presenta una concatenazione con il costrutto \text{then}, il quale specifica che il corpo della funzione a lui associato verrà eseguito solo quando la promessa precedente verrà risolta.\\
Lo stato di una \emph{promessa}, inoltre, stabilisce se questa è stata risolta senza errori oppure se qualche imprevisto è avvenuto. A questo serve lo stato \emph{rejected} di una particolare \emph{promessa}, il quale concretizza che sia avvenuto un errore. L'errore verrà quindi segnalato attraverso lo stato \emph{rejected} e tutti i costrutti che utilizzeranno quella \emph{promessa} saranno segnalati e dovranno gestire il caso particolare dell'errore.\\
Inolte è possibile concatenare più \emph{promesse} in modo controllato, così da creare una catena asincrona di esecuzione di funzioni, ma senza causare una diminuzione di chiarezza nel codice. In più, se una \emph{promessa} all'interno della catena solleva un'eccezione, sistematicamente l'errore viene gestito localmente o propagato al chiamante, con un maggior controllo nell'esecuzione.

\subsubsection{Promesse nel progetto}
Come presentato nella sezione precedente, le \emph{promesse} possono venir utilizzate per gestire in modo asincrono le chiamate \gls{ajax}. Ciò è vero e largamente utilizzato anche nel framework AngularJS, dove le \emph{promesse} sono parte integrante delle direttive \textbf{\$http} e \textbf{\$resource}. Infatti, queste direttive utilizzano gli stessi costrutti derivanti dalla libreria \textbf{Q} di Kriskowal\footcite{site:q}, sia per la risoluzione delle \emph{promesse} che per la loro gestione degli errori.\\
In particolare, la direttiva \textbf{\$http} mi è stata fondamentale per realizzare le richieste \gls{rest} e per la loro gestione tra \emph{servizi} e \emph{controller}. Qui sotto, un esempio di applicazione della direttiva \textbf{\$http} nel progetto di SkillMatrix.
\begin{verbatim}
expFactory.getExperiences = function(username) {
  return $http
    .get('/api/' + APP_VERSION.version + '/users/' + username + '/experiences');
};
\end{verbatim}

\subsection{Diagrammi di attività}
Vado ora ad illustrare i diagrammi di attività delle funzionalità principali di \emph{SkillMatrix}. Tali diagrammi fanno riferimento al linguaggio di modellazione \gls{uml}.

\subsubsection{Attività generali}
\begin{figure}[!h] 
    \centering 
    \includegraphics[width=0.8\columnwidth]{usecase/activity_dashboard} 
    \caption{Diagramma di attività generale di \emph{SkillMatrix}}
\end{figure}
Questo diagramma delle attività rappresenta la visione generale delle operazioni che si possono compiere all'interno dell'applicazione.\\
Dopo aver effettuato il \emph{login} con successo al sistema, si viene reindirizzati alla pagina del proprio profilo, con i propri dati aggiornati. Ora si può scegliere tra una lista di possibili menù:
\begin{itemize}
	\item esperienze professionali;
	\item titoli di studio e abilitazioni professionali (qualificazioni);
	\item skill;
	\item progetti associati;
	\item logout.
\end{itemize}
Le prime opzioni portano alla pagina di gestione corrispondente, mentre il pulsante di \emph{logout} consente di uscire dall'applicazione eliminando i dati di sessione.

\subsubsection{Attività di gestione delle skill}
\begin{figure}[!h] 
    \centering 
    \includegraphics[width=0.8\columnwidth]{usecase/activity_skill} 
    \caption{Diagramma di attività di gestione delle skill di \emph{SkillMatrix}}
\end{figure}
Questo diagramma delle attività è più specifico del precedente ed illustra le attività eseguibili nella pagina di gestione delle \emph{skill}.\\
Una volta giunti a questa pagina, si può decidere di filtrare la visualizzazione a seconda del livello di skill che si desidera visualizzare, sia questo \emph{Junior}, \emph{Intermediate} o \emph{Senior}, oppure una combinazione dei tre. La vista è aggiornata in \emph{real time}, dato che il filtraggio dei dati avviene in locale sulla macchina dell'utente, a seconda dei parametri selezionati.\\
Oltre alla visualizzazione, l'utente può inserire una nuova skill nel sistema. Qui, per poter inserire una skill valida, l'utente deve specificare:
\begin{itemize}
	\item Area, ad esempio \emph{Java};
	\item Ambito, ad esempio \emph{J2EE};
	\item Dominio, ad esempio \emph{Tomcat};
	\item Competenze, ovvero le versioni di competenza della skill;
	\item Livello, ovvero il livello di competenza della skill, ad esempio \emph{Junior}.
\end{itemize}
Dopo aver compilato i campi necessari per l'inserimento, l'utente può richiedere il salvataggio della skill. Se la skill non è considerata valida (perché manca qualche parametro oppure i dati inseriti non sono validi), l'applicazione segnala tali invalidità tramite dei messaggi \emph{popup} e richiede il corretto inserimento. Se invece l'utente seleziona il pulsante di \emph{reset}, l'utente può reinserire dei nuovi dati della skill. Se infine la skill è considerata valida, l'utente viene riportato alla pagina di gestione delle skill, aggiornata con la nuova skill appena inserita.