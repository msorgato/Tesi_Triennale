% !TEX encoding = UTF-8
% !TEX TS-program = pdflatex
% !TEX root = ../tesi.tex
% !TEX spellcheck = it-IT

%**************************************************************
\chapter{Introduzione}
\label{cap:introduzione}
%**************************************************************

Introduzione al contesto applicativo.\\

\noindent Esempio di utilizzo di un termine nel glossario \\
\gls{api}. \\

\noindent Esempio di citazione in linea \\
\cite{site:agile-manifesto}. \\

\noindent Esempio di citazione nel pie' di pagina \\
citazione\footcite{womak:lean-thinking} \\

%**************************************************************
\section{L'azienda}

Dal 1999 IKS focalizza il proprio agire per fornire ai Clienti soluzioni abilitanti per la realizzazione di servizi di e-business, nell’ambito della sicurezza e della \gls{governance} \gls{ict}, il tutto con un approccio rivolto all’ottimizzazione.\\
Sin dall’inizio, l’azienda investe su temi strategici per l’\gls{ict}: sicurezza informatica, automazione e governance di infrastrutture, sviluppo di soluzioni software complesse e tecnologicamente innovative, ottimizzazione di architetture informatiche di alto livello.\\
La propensione all’innovazione ed il costante investimento nella formazione permettono alla società a metà degli anni Duemila di affermarsi a livello nazionale con sedi a Padova, Milano e Roma e di diventare il riferimento tecnologico di aziende di alto livello che affidano all’infrastruttura \gls{ict} applicazioni e servizi business critical.\\
Grazie alla propria vocazione innovativa ed al continuo sviluppo, IKS oggi si presenta come un gruppo aziendale in grado di fornire supporto su temi che spaziano dalla \gls{compliance} all’ottimizzazione di processi di filiera, dal WEB 2.0 all’energy management.

\begin{figure}[H] 
    \centering 
    \includegraphics[width=0.3\columnwidth]{IKS_logo} 
    \caption{Logo di IKS}
\end{figure}

\subsection{Aree di intervento}
\paragraph{Sicurezza}
La priorità è garantire che le risorse e i servizi di un sistema informativo siano sempre accessibili, fruibili, garantendone però l’integrità e la conformità; proteggere l’azienda da violazioni e attacchi informatici accidentali e fraudolenti.

\paragraph{Ottimizzazione dell'infrastruttura}
In tempi di virtualizzazione, consolidamento e risparmio energetico ed in attesa del cloud computing è fondamentale ripensare al modello di infrastruttura con criteri legati alle modalità e tempistiche di provisioning dei sistemi, alla semplificazione dell’infrastruttura, al nuovo paradigma di sicurezza, al cost effective.

\paragraph{Governance}
La Governance è la risposta per l’azienda che oggi è chiamata a confrontarsi con la sempre maggiore complessità dei data center e delle applicazioni, la necessità di gestire i fenomeni dirompenti della virtualizzazione e dell’accesso a servizi in the cloud, l’obbligo stringente e vincolante di mantenere adeguati livelli di performance, nonché di ottimizzare i costi IT.

\paragraph{Ricerca e innovazione}
Progettare e costruire un’applicazione a valore aggiunto significa conoscere le possibilità e potenzialità della tecnologia. Il valore aggiunto è dato dal poter garantire competenze sempre aggiornate, continue attività di ricerca e di prototipizzazione su temi innovativi, capacità consolidate nel disegno di architetture complesse e di integrazione, forti partnership tecnologiche con i principali leader di mercato. Tutto ciò sostenuto da una forte, determinata quanto ragionevole propensione all’innovazione.

%**************************************************************

\section{L'idea}

Il progetto \myTitle{} nasce come strumento di gestione interna delle competenze dei dipendenti e dei progetti presenti in azienda. L'obiettivo di tale progetto è una migliore organizzazione del personale dedicato ai vari progetti, con una conseguente maggiore efficienza ed efficacia nell'assegnazione dei ruoli nei processi ed attività. Fino ad ora, all'interno di IKS, tale compito veniva svolto dalla consultazione e modifica di un foglio elettronico \emph{Excel}. Questa soluzione, oltre ad essere poco pratica e difficilmente manutenibile, ha portato all'esigenza di uno strumento più specifico ed espandibile. Da qui la creazione del progetto \myTitle{}.\\
Lo strumento \myTitle{} è pensato per essere utilizzato da diverse tipologie di utenti, dal singolo sviluppatore al project manager.\\
L'applicazione sarà composta da un \gls{frontend} realizzato in \emph{AngularJS}, un \gls{backend} realizzato tramite \gls{spring} e un database di persistenza. L'attività di Stage ha riguardato la realizzazione del frontend e la definizione delle \gls{api} \gls{rest} del backend.\\
L'utente finale sviluppatore può vedere \myTitle{} come una sorta di curriculum digitale ampliato, in cui poter inserire le proprie competenze riconosciute, le esperienze acquisite ed i progetti a cui ha preso parte. Nella propria \emph{dashboard}, l'utente finale può gestire le proprie informazioni e richiedere l'inserimento di nuove entità, come nuovi progetti a cui ha preso parte, nuove skill acquisite, certificazioni esterne o interne o altri titoli formativi e le sue esperienze lavorative pregresse.\\ 
Chi ha compiti di gestione (ad esempio la figura del Responsabile di progetto), utilizza lo strumento per vedere quali risorse può utilizzare per un dato progetto. In base alle competenze richieste può quindi consultare quali risorse può richiedere per svolgere il progetto con maggior efficienza ed efficacia.\\ 
Si prefissa quindi come obiettivo la creazione di un portale aziendale unificato per una migliore gestione del personale ed il suo dislocamento all’interno dei progetti aziendali, oltre che alla gestione dei curricula vitae del personale aziendale.\\
Per quanto riguarda l'organizzazione del progetto, è stato deciso che \myTitle{} avrebbe avuto due layer \emph{Frontend} e \emph{Backend} comunicanti tramite richieste \gls{rest}. Questo per rendere le due componenti più separate possibile, dovendo soltanto definire le \gls{api} di comunicazione tra il client ed il server. Inoltre, essendo che questo progetto di stage ha coperto la parte \emph{Frontend}, in un progetto futuro di implementazione del \emph{Backend}, il server dovrà solamente esporre le corrette funzionalità alla chiamata delle \gls{api} \gls{rest} concordate.




%**************************************************************

\section{Vincoli}
Questa sezione illustra i vincoli che sono stati stipulati all'inizio dell'attività di stage, sia tecnici che organizzativi.

\subsection{Tecnologici}
I vincoli sullo stack tecnologico da utilizzare nell'implementazione di \myTitle{} sono stati fissati sin dai primi contatti. Per questo, la prima settimana di stage è stata rivolta all'apprendimento di tali strumenti e all'integrazione con gli strumenti e le procedure aziendali.\\
Tali strumenti sono:
\begin{itemize}
	\item \emph{AngularJS}, framework JavaScript per la scrittura di applicazioni web frontend;
	\item \emph{Bootstrap}, libreria \gls{css} grafica per gestire la presentazione di pagine web \gls{html};
	\item \emph{JasmineJS}, framework JavaScript utilizzato per la scrittura di test, soprattutto di unità ed \gls{e2e}. 
\end{itemize}
La scelta di queste tecnologie ovviamente non è casuale. Nello sviluppo di applicazioni web con \emph{AngularJS}, lo stack tecnologico utilizzato più diffusamente è proprio questo. L'utilizzo di \emph{Jasmine} è quasi obbligatorio, data la sua immediatezza di configurazione con \emph{AngularJS} e la filosofia stessa del framework frontend, che è stato pensato appunto per rendere il testing più semplice.\\
\emph{Bootstrap}, d'altro canto, è estremamente facile da integrare con qualsiasi strumento che si basi su \gls{html} per fornire le sue viste.

\subsection{Temporali}
Per la realizzazione di questo progetto sono state impiegate le 8 settimane di lavoro di 40 ore/settimana. Il periodo in cui si è svolto ha compreso le settimane che sono intercorse tra il 20 Aprile e il 13 Giugno, per un totale di ore compreso tra le 300 e 320 stabilite all'inizio della pianificazione.\\
Di queste ore, le prime 40 sono state dedicate alla mia formazione sulle tecnologie adottate nel progetto e nell'istruzione sull'utilizzo degli strumenti aziendali, per poi iniziare la progettazione e lo sviluppo vero e proprio del progetto.

%**************************************************************


\section{Organizzazione del testo}

\begin{description}
    \item[{\hyperref[cap:processi-metodologie]{Il secondo capitolo}}] descrive ...
    
    \item[{\hyperref[cap:descrizione-stage]{Il terzo capitolo}}] approfondisce ...
    
    \item[{\hyperref[cap:analisi-requisiti]{Il quarto capitolo}}] approfondisce ...
    
    \item[{\hyperref[cap:progettazione-codifica]{Il quinto capitolo}}] approfondisce ...
    
    \item[{\hyperref[cap:verifica-validazione]{Il sesto capitolo}}] approfondisce ...
    
    \item[{\hyperref[cap:conclusioni]{Nel settimo capitolo}}] descrive ...
\end{description}

Riguardo la stesura del testo, relativamente al documento sono state adottate le seguenti convenzioni tipografiche:
\begin{itemize}
	\item gli acronimi, le abbreviazioni e i termini ambigui o di uso non comune menzionati vengono definiti nel glossario, situato alla fine del presente documento;
	\item per la prima occorrenza dei termini riportati nel glossario viene utilizzata la seguente nomenclatura: \emph{parola}\glsfirstoccur;
	\item i termini in lingua straniera o facenti parti del gergo tecnico sono evidenziati con il carattere \emph{corsivo}.
\end{itemize}