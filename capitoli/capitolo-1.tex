% !TEX encoding = UTF-8
% !TEX TS-program = pdflatex
% !TEX root = ../tesi.tex
% !TEX spellcheck = it-IT

%**************************************************************
\chapter{Introduzione}
\label{cap:introduzione}
%**************************************************************

Introduzione al contesto applicativo.\\

\noindent Esempio di utilizzo di un termine nel glossario \\
\gls{api}. \\

\noindent Esempio di citazione in linea \\
\cite{site:agile-manifesto}. \\

\noindent Esempio di citazione nel pie' di pagina \\
citazione\footcite{womak:lean-thinking} \\

%**************************************************************
\section{L'azienda}

Descrizione dell'azienda.

%**************************************************************
\section{L'idea}

Il progetto \myTitle{} nasce come strumento di gestione interna delle competenze e dei progetti presenti in azienda. L'obiettivo di tale progetto è una migliore organizzazione del personale dedicato ai vari progetti, con una conseguente maggiore efficienza ed efficacia nell'assegnazione dei ruoli nei processi ed attività. Fino ad ora, all'interno di IKS, tale compito veniva svolto dalla consultazione e modifica di un foglio elettronico \emph{Excel}. Questa soluzione, oltre ad essere poco pratica e difficilmente manutenibile, ha portato all'esigenza di uno strumento più specifico ed espandibile. Da qui la nascita del progetto \myTitle{}.\\
L'applicazione sarà composta da un \gls{frontend} realizzato in \emph{AngularJS}, un \gls{backend} realizzato tramite \emph{Spring} e un database di persistenza. L'attività di Stage ha riguardato la realizzazione del frontend e la definizione delle \gls{api} \gls{rest} del backend. 
%definisci SPRING

Il personale aziendale inserisce le proprie competenze nel sistema, selezionate a partire da un elenco precompilato. 
Chi ha compiti di gestione utilizza lo strumento per vedere quali risorse può utilizzare per un
progetto. Ogni utente potrà inoltre tenere traccia dei vari progetti a cui ha collaborato, delle competenze acquisite e dei vari attestati ottenuti durante la carriera aziendale e non.
Si prefissa quindi come obiettivo la creazione di un portale aziendale unificato per una migliore gestione del personale ed il suo dislocamento all’interno dei progetti aziendali, oltre che alla gestione dei curricula vitae del personale aziendale.


%**************************************************************
\section{Organizzazione del testo}

\begin{description}
    \item[{\hyperref[cap:processi-metodologie]{Il secondo capitolo}}] descrive ...
    
    \item[{\hyperref[cap:descrizione-stage]{Il terzo capitolo}}] approfondisce ...
    
    \item[{\hyperref[cap:analisi-requisiti]{Il quarto capitolo}}] approfondisce ...
    
    \item[{\hyperref[cap:progettazione-codifica]{Il quinto capitolo}}] approfondisce ...
    
    \item[{\hyperref[cap:verifica-validazione]{Il sesto capitolo}}] approfondisce ...
    
    \item[{\hyperref[cap:conclusioni]{Nel settimo capitolo}}] descrive ...
\end{description}

Riguardo la stesura del testo, relativamente al documento sono state adottate le seguenti convenzioni tipografiche:
\begin{itemize}
	\item gli acronimi, le abbreviazioni e i termini ambigui o di uso non comune menzionati vengono definiti nel glossario, situato alla fine del presente documento;
	\item per la prima occorrenza dei termini riportati nel glossario viene utilizzata la seguente nomenclatura: \emph{parola}\glsfirstoccur;
	\item i termini in lingua straniera o facenti parti del gergo tecnico sono evidenziati con il carattere \emph{corsivo}.
\end{itemize}