% !TEX encoding = UTF-8
% !TEX TS-program = pdflatex
% !TEX root = ../tesi.tex
% !TEX spellcheck = it-IT

%**************************************************************
\chapter{Processi e metodologie}
\label{cap:processi-metodologie}
%**************************************************************

\intro{Brevissima introduzione al capitolo}\\

%**************************************************************
\section{Processo sviluppo prodotto}

\subsection{Agile}
Agile è una metodologia di sviluppo nata in contrapposizione ad altri modelli più stringenti e formali, quali ad esempio il modello a Cascata o a Spirale.\\
I principi generali dell’Agile Programming sono descritti nell’Agile Manifesto\footcite{site:agile-manifesto}, e possono essere riassunti in quattro punti cardine:
\begin{itemize}
	\item \emph{individui e interazioni:} organizzazione e motivazione autonoma sono importanti, come lo sono le interazioni personali come la condivisione dello stesso luogo di sviluppo;
	\item \emph{software funzionante:} un prodotto che funziona è più utile e meglio accettato di documenti cartacei presentati agli acquirenti durante i meeting;
	\item \emph{collaborazione con gli acquirenti:} i requisiti non possono essere pienamente individuati all’inizio del ciclo di sviluppo del software. Perciò l’interazione con i clienti e gli stakeholder è estremamente importante;
	\item \emph{responsività al cambiamento:} i metodi agile sono focalizzati sul fornire risposte veloci al cambiamento e allo sviluppo continuo.
\end{itemize}
Lo sviluppo Agile permette di valutare ed eventualmente correggere la direzione durante il processo stesso. Questo risultato è ottenuto attraverso brevi e regolari iterazioni di lavoro, alla fine delle quali ogni team deve presentare un incremento del prodotto, considerandolo come una nuova feature applicabile e funzionante. Concentrandosi sulla ripetizione di cicli di lavoro brevi e definiti, proporzionati alla funzionalità da consegnare, le metodologie Agile si definiscono “iterative” o “incrementali”. Nel modello a cascata, i team di sviluppo hanno una sola opportunità di realizzare un aspetto del progetto nel modo giusto. Nel paradigma Agile, ogni aspetto dello sviluppo - requisiti, progettazione, ecc. - è continuamente rivisitato. Quando un team si ferma e rivaluta la direzione di un progetto ogni due settimane, è possibile cambiare tale direzione con facilità.\\
Questo approccio “ispeziona-e-adatta” riduce i costi e il tempo di consegna dello sviluppo. Dato che i vari team possono sviluppare software nello stesso tempo in cui individuano i requisiti, la “paralisi dell’analisi” ha meno probabilità di bloccare i progressi di un team. E siccome il ciclo di lavoro di un team è limitato a due settimane, gli stakeholder hanno opportunità ricorrenti di analizzare i rilasci del software e il loro feedback dal mercato.\\
La metodologia Agile si basa sul concetto di \emph{User Story}, ovvero un compito significativo che l'utente vuole poter svolgere attraverso il software realizzato. Le User Stories catturano il 'chi', 'cosa' e 'perché' di un requisito in maniera semplice e concisa.

\subsubsection{Kanban}
%storia
Kanban è un framework usato per implementare la metodologia agile. Negli anni ‘40, Toyota ottimizzò i suoi processi modellandoli come se fossero degli scaffali in un supermercato. I supermercati offrono una quantità di prodotti atti a soddisfare con il minimo spreco la domanda dei consumatori. Siccome i livelli di inventario sono conseguenti ai pattern di consumazione, il supermercato ottiene una significante efficienza ed ottimizzazione nella gestione dell’inventario.\\
Quando Toyota portò questa idea ai suoi piani di lavoro, i team (come ad esempio il team che aggiunge le portiere al telaio dell’auto) consegnavano una carta, o “kanban”, agli altri dipendenti (ad esempio, ai team che assemblano le portiere) per segnalare di aver ecceduto la capacità e di essere pronti a ritirare più materiale. Anche se la tecnologia di segnalazione si sia evoluta, questo sistema è ancora al centro della produzione “just in time”.\\
Kanban presenta lo stesso comportamento per i team software. Tramite la comparazione dell’ammontare del lavoro in progresso rispetto alla capacità del team, kanban fornisce agli stessi opzioni di pianificazione più flessibili, output più veloci, miglior concentrazione sui singoli compiti e trasparenza durante il ciclo di sviluppo.\\
%metodologia
Kanban è un metodo di gestione della consapevolezza del lavoro con un’enfasi particolare sulla consegna “just in time”, assicurandosi nel mentre che i membri del team non abbiano troppo lavoro rispetto alle loro capacità di carico. In questo approccio, il processo, dalla definizione dei task alla consegna al cliente, è mostrato visivamente ai partecipanti. I membri del team estraggono ogni unità di lavoro da una coda.
Kanban nel contesto dello sviluppo software può essere inteso come un sistema di gestione di processo visuale, che dice cosa produrre, quando e in quale quantità produrlo - ispirato dal sistema di produzione Toyota.\\
Nella sua forma più basilare, un sistema di kanban consiste di una grande lavagna appesa ad un muro con delle carte o dei postit organizzati in colonne con dei numeri su ogni colonna\footcite{site:kanban}.\\
Le carte rappresentano le unità di lavoro mentre attraversano il processo di sviluppo, rappresentato dalle colonne.\\
I limiti sono la differenza sostanziale tra una lavagna Kanban e un’altra storyboard qualsiasi. Limitando l’ammontare di Work-In-Progress (WIP) ad ogni passo del processo, si previene la sovrapproduzione di risorse e si rivelano dinamicamente i colli di bottiglia, così che possano essere presi dei provvedimenti adattativi il prima possibile.\\
Ogni team può quindi avere una determinata quantità di lavoro in esecuzione per unità di tempo. Così facendo si limitano gli sprechi di tempo dovuti al cambio di contesto chiesto se si implementa un certo numero di User Stories parallelamente. Ogni membro del team, ogniqualvolta finisce uno step del processo di realizzazione di una User Story, sposta la scheda corrispondente nella colonna successiva della kanban, la quale proseguirà nel suo processo, mentre lo sviluppatore potrà passare all’implementazione di una nuova User Story.

\paragraph{Procedure}
Le procedure seguite nell'applicazione della metodologia Kanban si appoggiano ai doftware aziendali rilasciati da Atlassian, ovvero JIRA e STASH.\\
In particolare, l'implementazione di una singola User Story passa attraverso diversi passi definiti.\\
Dal lato organizzativo della stesura e realizzazione delle User Stories e della Kanban vera e propria, i passi da seguire sono stati i seguenti, applicati al framework JIRA:
\begin{enumerate}
	\item il Responsabile di progetto scrive la User Story;
	\item il Responsabile decide a chi assegnare l'implementazione della User Story;
	\item lo sviluppatore designato prende in carico il ticket aperto, contrassegnandolo come "In Progress";
	\item lo sviluppatore realizza la funzionalità descritta nella User Story della comanda;
	\item lo sviluppatore notifica l'avvenuta implementazione della funzionalità contrassegnando il ticket come "Done";
	\item il Responsabile viene notificato dell'avvenuta realizzazione e provvede all'inserimento della nuova funzionalità nel progetto.
\end{enumerate}
A questa procedura si associa anche la gestione vera e propria dei sorgenti aziendali, memorizzati in repository acceduti tramite il framework STASH. L'implementazione di ogni feature prevede il passaggio per vari punti, necessari alla stabilità e all'organizzazione del sistema.\\
La procedura da seguire all'interno del framework STASH è la seguente:
\begin{enumerate}
	\item prendere in carico una User Story dal backlog;
	\item creare un nuovo branch dall'attuale branch \emph{develop} del repository;
	\item implementare la funzionalità presa in carico, ricordandosi di effettuare commit periodici con messaggi esplicativi del lavoro effettuato;
	\item al termine dell'implementazione, assicurarsi che i test disegnati per le unità implementate passino, così da non introdurre errori di unità nel sistema;
	\item una volta appurato che i test sono positivi, proseguire con la richiesta di pull nel branch \emph{develop} da notificare al Responsabile del progetto.
\end{enumerate}
A questo punto si può verificare un imprevisto, ovvero che nel tempo in cui viene implementata una User Story, il branch \emph{develop} subisca delle modifiche. Questo fatto comporta una pull del \emph{develop} nel branch attuale ed una conseguente risoluzione de conflitti che potrebbero venirsi a creare. Dopodiché si prosegue con la procedura:
\begin{enumerate}[resume]
	\item esaminare la risposta del Responsabile; in caso di approvazione, la procedura termina;
	\item in caso di mancata approvazione della pull request, esaminare i punti in cui vengono alzate obiezioni da parte del Responsabile apportare le dovute modifiche al codice;
	\item ritornare al punto 4 e iterare fino ad avvenuta approvazione della pull request.
\end{enumerate}


\subsection{Test Driven Developement}
Il modello di sviluppo Test Driven (guidato dai test) prevede che lo sviluppo di ogni unità software avvenga soltanto dopo aver scritto il corrispondente test sulla base delle specifiche dell’unità stessa.\\
Il dovere di ogni programmatore che segue questo modello è quello di scrivere il minimo quantitativo di codice necessario a far validare il test, minimizzando gli sprechi e provando la correttezza del prodotto, delegando il refactoring stilistico ad un momento successivo, in cui si ha già un codice testato e funzionante.\\
Il motto dello sviluppo Test Driven è appunto “Red, Green, Refactor”; la procedura da seguire prevista da questa metodologia è la seguente:
\begin{enumerate}
	\item scrivere un “singolo” test di unità che descrive una funzionalità del programma;
	\item eseguire il test, il quale dovrebbe fallire, dato che il programma non presenta ancora la feature descritta (da qui, la fase Rossa);
	\item scrivere la quantità minima di codice necessaria far passare il test (fase Verde);
	\item eseguire il Refactoring del codice, in modo da eliminare eventuali ridondanze e superficialità;
	\item ripetere, accumulando test di unità con il passare dello sviluppo.
\end{enumerate}
Il framework AngularJS è stato creato appositamente per rendere il testing più efficiente e semplice possibile, grazie a molte funzionalità focalizzate al testing già comprese all’interno del framework. Questa scelta è stata compiuta perché JavaScript ha una grandissima potenza espressiva, ma quasi nessun controllo da parte del compilatore, il che rende necessaria la costante presenza dei test nello sviluppo con questo linguaggio.
